% Discussion about forming glossary entry, singular and plural, first and second appearance
% https://tex.stackexchange.com/questions/128396/defining-plural-form-for-glossaries-entries
% call with: \gls{key} & \glspl{key}
%
% Wikibooks Page: https://en.wikibooks.org/wiki/LaTeX/Glossary
% CTAN source: https://ctan.org/pkg/glossaries?lang=en

% EXAMPLE GLOSSARY
%\newglossaryentry{LED}{%
%  name={LED},                                      ← Glossary appearance
%  description={light emitting diode},              ← Glossary description
%  first={\glsentrydesc{LED} (\glsentrytext{LED})}, ← First appearance normal
%  firstplural={\glsentrydescplural{LED} (\glsentryplural{LED})} ← First appearance plural
%  text={LED},                                      ← text appearance 2nd … time (`name` by default)
%  plural={LEDs},                                   ← plural appearance after 1st
%  descriptionplural={light emitting diodes}        ← plural description
%}

% EXAMPLE ACRONYM
%\newacronym[plural=cM,firstplural=centiMorgans (cM)]{cM}{cM}{centiMorgan}

\newglossaryentry{sdr}{%
        name={Software Defined Radio (SDR)},
        description={Software defined radios are general purpose radio
        solutions capable to adapt to the current
        situation. Therefore, important parameters (e.g. coding scheme
        of the data-stream) for transmitter and receiver can be
        configured by the software and/or programmable hardware.},
        first={Software Defined Radio (SDR)},
        firstplural={Software Defined Radios (SDR)},
        text={SDR},
        plural={SDRs},
        }
\newglossaryentry{glonass}{%
        name={Globalnaya Navigatsionnaya Sputnikovaya Sistema (GLONASS)},
        description={Satellite based navigation system operated by Roscosmos.},
        first={Global Navigation Satellite System (GLONASS)},
        text={GLONASS}
        }
\newglossaryentry{gnssr}{%
        name={Global Navigation Satellite System-Reflectometry (GNSS-R)},
        description={Radiometric measurement based on forward or
        back-scattered radio waves from global navigational satellite
        systems.},
        first={Global Navigation Satellite System-Reflectometry (GNSS-R)},
        text={GNSS-R},
        }
\newglossaryentry{zenit}{%
        name={zenit},
        description={Straight above, normal vector opposite of Gravity.},
        first={zenit},
        text={zenit}
        }
\newglossaryentry{nadir}{%
        name={nadir},
        description={Straight below, normal vector direction of Gravity.},
        first={nadir},
        text={nadir}
        }
\newglossaryentry{leo}{%
        name={Low Earth Orbit (LEO)},
        description={Satellite orbit at 200-2,000\,km altitude.},
        first={Low Earth Orbit (LEO)},
        text={LEO}
        }
\newglossaryentry{meo}{%
        name={Medium Earth Orbit (MEO)},
        description={Satellite orbit at 2,000-35,786\,km altitude.},
        first={Medium Earth Orbit (MEO)},
        text={MEO}
        }

\newacronym{cots}{COTS}{Commercial off-the-shelf}
\newacronym[plural=DDMs, firstplural=Delay-Doppler-Maps (DDM)]{ddm}{DDM}{Delay-Doppler-Map}
\newacronym[plural=FFTs, firstplural=Fast Fourier Transforms (FFT)]{fft}{FFT}{Fast Fourier Transform}
