\section{Formatting requirements and thesis template}
\label{Introduction} %Allows referencing titles with \ref

This document presents author guidelines and formatting requirements
for thesis defended at the Faculty of Information Technology at
Tallinn University of Technology. These requirements are obligatory
for all bachelor’s and master’s thesis (further referred as
thesis). Sections 1 and 2 of this document constitute formatting
requirements whereas Sections 3 and 4 provide helpful hints for word
processing with MS Word and LibreOffice Writer
correspondingly. Authors are strongly advised to familiarize with
Sections 1 and 2 before starting to write thesis. This document is
formatted according to the requirements declared in this document, and
acts as an example of correct formatting.

\subsection{Authors’ toolkit}

Authors are encouraged to download and use the templates prepared to
ensure correct formatting of thesis. Templates are available for Word
processing programs MS Word (version 2013 and newer) and LibreOffice
Writer (version 5 and newer). The authors’ toolkit is downloadable as
a single archive file in ZIP-format.

The toolkit contains guidelines and templates to compose thesis in
Estonian or in English. Authors should download the toolkit file, and
unzip it to their computer to a folder of their choice. The toolkit
archive contains the following:

\begin{itemize}
\item  Author guidelines and formatting requirements document (PDF)
\item \textbf{MS Word templates:}
  \begin{itemize}
  \item Template with predefined styles, thesis
    structure and title page \\
    (FIT\_template\_with\_structure\_ENG.dotx);
  \item Template with predefined styles and title \\
    page only (FIT\_template\_ENG.dotx);
  \end{itemize}
\item \textbf{LibreOffice templates:}
  \begin{itemize}
  \item Template with predefined styles, thesis
    structure and title page \\
    (FIT\_template\_with\_structure\_ENG.ott).
  \end{itemize}
\end{itemize}

To create a new text document based on the template authors should
open it by double- clicking on the template file name. Opening
template file with the Open command opens the template for editing.

\subsection{General formatting requirements}

Thesis must be formatted on an A4 page (210 x 297 mm), with margins
from left and right set to 30 mm, and top and bottom 25 mm. The
content text layout is in one column.

The thesis \textbf{font} is \textit{Times New Roman} 12 pt
\textit{Regular}. Correct formatting of text is ensured with the use
of the style \textit{Body Text} that has the following properties:
\textit{Times New Roman} 12 pt \textit{Regular}, line spacing 1.5,
spacing after 12 pt, text alignment \textit{Justify}. The style
\textit{Body Text} is based on style \textit{Normal}. Table 1 lists
all the requirements for different document parts and styles available
in the template to ensure correct formatting.

Words or sentences which need highlighting can be presented as
double-spaced, in \textbf{Bold} or \textit{Italic} (e.g., expressions
in foreign language).

The font colour of the paragraphs and headings is black. The use of
colours on figures is not limited.  The content of the thesis is
structured into sections using up to three levels of headings. Adding
a new level is justified if it contains more than one paragraph of
text. The use of level 4 headings (\textit{Heading 4}) must be
avoided. First level headings (\textit{Heading 1}) always start a new
page. Headings are numbered using Arabic numerals. Multilevel headings
contain a reference to higher level heading, whereas the numbers are
separated by a dot.

As an exception the following headings are not numbered: Author’s
declaration of originality, Abstract
(\textit{Annotatsioon},\textcolor{red}{[Russian is not supported in
  this document yet!]}  %\textit{Aннотация}), List of abbreviations and
terms, Table of contents, List of figures, List of tables, References,
Appendixes. Headings for appendixes always start with a word
\textit{Appendix}, which is followed by a number (or a letter), no
automatic numbering is applied.

\textbf{Correct formatting of headings} is ensured with the use of
styles available in the template: \textit{Heading 1}, \textit{Heading
  2}, and \textit{Heading 3}.

\subsection{Formatting requirements and styles available in templates}

To ensure correct formatting of thesis and help authors in this
process the templates contain different style definitions applicable
for different document parts. These styles are based on the style
\textit{Normal}: \textit{Times New Roman} 12 pt, \textit{Regular},
space before and after 0 pt, spacing \textit{Single}. Table
\ref{Tab:formatting1} lists document parts and styles defined for
their correct formatting together with the requirements.



\begin{table}[ht]
  \caption{Formatting requirements and applicable styles in the templates}
  \label{Tab:formatting1}
  \centering
  % sum of the lengths should be 210 - 60 mm = 150
  \begin{tabular}{|L{3.7cm}|L{3.0cm}|L{7.2cm}|}
    \hline
    \textbf{Document part} & \textbf{Style name} & \textbf{Settings}\\  \hline
    Thesis code      &  \textit{Centred} &  Times New Roman 12 pt, centred\\ \hhline{-~~}
    Author’s details & & In further referred as Lettering. \\\hhline{-~~}
    Thesis type      & &  \\ \hline
    Supervisor’s details & \textit{Normal} & Lettering 12 pt, align left \\ \hline
    Thesis title & \textit{Heading\_title} & \textit{Times New Roman} 20 pt, Bold, All Caps,
                                             centered \\ \hline

    Thesis body text & \textit{Body Text} & Times New Roman 12 pt, line spacing 1.5, spacing after 12 pt, justified\\ \hline
    Headings:

    --References

    --Appendixes  & \textit{Heading\_unnumber} & Lettering 16 pt, Bold, spacing before 60 pt,
                                                 after 18 pt, align left. Starts a new page. No
                                                 numbering.\\ \hline

    Headings:

    --Table of contents
    --Author’s declaration of
    originality

    --Annotatsioon
    --Abstract

    --List of abbreviations
    and terms

    --List of figures

    --List of tables & \textit{Heading\_center}& Lettering 16 pt, Bold, spacing before 60 pt,
                                                 after 18 pt (like Heading 1). Starts a new page.
                                                 No numbering. \\ \hline
    Level 1 heading & \textit{Heading 1} & Lettering 16 pt, Bold, align left, spacing before 60 pt, after 18 pt, numbered with Arabic
                                           numerals, space between number and heading
                                           text, starts a new page.
    \\ \hline
  \end{tabular}
\end{table}

\begin{table}[!ht]
  \centering
  \begin{tabular}{|L{3.7cm}|L{3.0cm}|L{7.2cm}|}
    \hline
    \textbf{Document part} & \textbf{Style name} & \textbf{Settings}\\  \hline

    Level 2 heading & Heading 2 & Lettering 14 pt, Bold, align left, spacing before
                                  24 pt, after 12 pt, numbering with Arabic
                                  numerals and a reference to higher level
                                  heading with a dot between numbers (e.g. 1.1) \\ \hline
    Level 3 heading & Heading 3 &Lettering 12 pt, Bold, align left, spacing before
                                  and after 12 pt, numbering with Arabic
                                  numerals and a reference to higher level
                                  heading with a dot between numbers
                                  (e.g. 1.1.1) \\ \hline
    Table header &Table\_head & Lettering 11pt, Bold, align left, line spacing
                                1.15, space before and after 3 pt \\ \hline

    -- Table text

    -- Terms in the list of
    abbreviations & Table\_text & Lettering 11 pt, align left, line spacing 1.15,
                                  space before and after 3 pt \\ \hline

    Figure/Table caption
    single line & Caption & Lettering 10 pt, line spacing 1.15
                            space before and after 6 pt, centered \\ \hline

    Figure/Table caption
    multiline & Caption\_multiline & Lettering 10 pt, line spacing 1.15
                                     space before and after 6 pt, Justify \\ \hline

    Program code & Program\_code & Consolas, 11 pt, single line spacing,
                                   space before and after 2 pt, align left \\ \hline

    Equations

    (on separate line) & Equation & Lettering 12 pt, single line spacing, space before and after 12 pt \\ \hline

    Bulleted list & List Bullet & Lettering 12 pt, square bullet, line spacing 1.5;
                                  bullet indentation 0.63 cm; spacing 12 pt,
                                  Justify, no spacing between lines of bulleted
                                  list \\ \hline

    Numbered list & List Number & Lettering 12 pt, line spacing 1.5; space 12 pt,
                                  indentation 0.63 cm; Justify, no spacing
                                  between lines of numbered list \\ \hline

    Items in the list of
    abbreviations and terms & Normal or
                              Table\_text & Lettering 12 pt, line spacing 1.5; spacing 0 \\ \hline

    References & References & Lettering 11 pt, line spacing 1.15, spacing
                              11 pt \\ \hline

    Footnotes& Footnote Text &
                               Lettering 10 pt, line spacing 1.15, spacing
                               before and after 0 pt, restart numbering for
                               each page \\ \hline

  \end{tabular}
\end{table}


\begin{table}[!ht]
  \centering
  \begin{tabular}{|L{3.7cm}|L{3.0cm}|L{7.2cm}|}
    \hline
    \textbf{Document part} & \textbf{Style name} & \textbf{Settings}\\  \hline
    Figure &  Figure & Centered. The style adds a frame and necessary space around \\ \hline

    References &list & List\_bibliogr Times New Roman 11 pt, numbered with
                       numbers in square brackets, spacing before and
                       after 2 pt \\ \hline

    Row in Table of
    contents: & & \\

    -- Level 1

    -- Level 2 & TOC1

                 TOC2  &  Lettering 12 pt, line spacing 1.5; align left

                         Lettering 12 pt, line spacing 1.5;
                         indentation 0.42 cm \\

    -- Level 3 &  TOC3 &  Lettering 12 pt, line spacing 1.5;
                         indentation 0.85 cm \\ \hline
  \end{tabular}
\end{table}
\subsection{Thesis volume}

The volume of bachelor’s thesis main content should remain in the
range of 25 to 35 pages, and for master’s thesis in between 30 to 60
pages. The main content does not include the title page, abstract,
table of contents, list of references and appendices.

Despite the given ranges, the number of pages of the main content of
the thesis should be as large as necessary and as small as
possible. The text must be presented concisely and the excess number
of pages is permissible only in well-justified cases.
