\section{[Chapter Title]}

This chapter provides some essential \LaTeX examples (lists, tables,
etc.) and hints.

\subsection{Hints for text processing}

Here are some tips for using \LaTeX:

\begin{itemize}
\item Words are separated by one or       more space characters between
  them. If a word is followed by a punctuation mark, punctuation should
  follow the word immediately, and a space character should be added
  after punctuation. Redundant
  spaces      are
  automatically             ignored.

\item Text is automatically divided into lines. The division of text
  into paragraphs is done by using two newlines.

\item Sometimes it is necessary to keep words or symbols together so
  that they will not be distributed on separate lines, for example
  initials of given name and surname. To add such a non-breaking space
  use \textasciitilde\ character (example: Kersti~Kaljulaid).

\item To prevent other words from breaking you can use mbox. Note
  that using \mbox{non-breaking} hyphen from extdash package is also
  possible, but it will not prevent the same word to be broken
  elsewhere.

\item \textbf{A dash} (--) is longer than hyphen and it is not just
  the minus symbol. Use two minus characters to insert a dash
  (example: --).

% TODO ↓ what's the LaTeX equivalent for this?
\item Hyphenation should not be something to worry about. If you want to hyphenate a
  word manually, just add Ctrl+Hyphen.

% TODO ↓ is this correct?
\item To break a long line (e.g., title), use double backslash (example: hello\\world).
\end{itemize}

\subsection{Unordered list (example)}

Unordered list example:
\begin{itemize}
\item Item in unordered list
\item Deeper unordered list:
  \begin{itemize}
  \item \textbf{Foo}
  \item \textit{Bar}
  \end{itemize}
\end{itemize}

\subsection{Citation (example)}

Example of citation \cite{urlSource}.

\newpage
\subsection{Table (example)}

\begin{table}[!h]
  \caption{This is a table caption}
  \centering
  \begin{tabular}{|L{5.1cm}|L{9.1cm}|}
    \hline
    \textbf{Column 1} & \textbf{Column 2}\\ \hline
    One row     & Some text\\ \hline
    Another row & Multi-line cell

                  More text here\\ \hline

    Foo & Text here\\ \hhline{-~}
    Bar & (more more)\\ \hhline{-~}
    Baz & \\ \hline
  \end{tabular}
\end{table}

\newpage
\subsection{Fancy Table (example)}

% TODO take some example from Siavoosh

\subsection{Figures (example)}

% TODO use public domain images

\begin{figure}[!ht]
  \centering
  \includegraphics[width=0.8\textwidth]{figures/TTU_peamine_logo_EST_print}
  \caption{ Example: TUT main logo in Estonian}
  \label{fig:logo}
\end{figure}

Figure \ref{fig:logo} presents an example of figure with short single
line caption.

\begin{figure}[!ht]
  \centering
  \includegraphics[width=0.8\textwidth]{figures/TTU_alternatiivne_logo_EST_ENG_print}
  \caption{Example: Lorem ipsum dolor sit amet, consectetuer
    adipiscing elit. Etiam lobortis facilisis sem. Nullam nec mi et
    neque pharetra sollicitudin. Praesent imperdiet mi nec ante.
  }
  \label{fig:logo2}
\end{figure}

Figure \ref{fig:logo2} presents an example of a figure with long multi-line caption.


\begin{figure}[!ht]
  \centering
  \begin{subfigure}[b]{0.45\textwidth}
    \includegraphics[width=\textwidth]{figures/TTU_peamine_logo_ENG_print}
    \caption{}
  \end{subfigure}
  \qquad
  \begin{subfigure}[b]{0.45\textwidth}
    \includegraphics[width=\textwidth]{figures/ttu_peamine_logo_eng_must-valge_negatiivis}
    \caption{}
  \end{subfigure}
  \caption{Example: TUT main logo in: (a) English, (b) negative black and white}
  \label{fig:logo3}
\end{figure}

Figure \ref{fig:logo3} presents an example of a multi-part figure.

\subsection{Program code}

Figure \ref{alg:example} presents an example of formatted program
code.

\begin{algorithm}[!ht]
  Public Function computeSomething() \\
  Dim i, j As Integer\\
  \For {i = 1 To 10}{
    \For {j = 1 To 10}{
      Do something in loop\\
      Next j\\
    }
    Next i\\
  }
  Return i + j\\
  End Function\\
  % use \captionof since algorithm should be declared as figure!
  \captionof{figure}{How to write algorithms}\label{alg:example}
\end{algorithm}


\subsection{Mathematical expressions and equations (example)}

Equation \ref{eq:equation-example} presents an example of correct
formatting and referencing.

\begin{equation} \label{eq:equation-example}
  (x+a)^n = \sum_{k=0}^{n} {n \choose k} x^ka^{n-k}
\end{equation}


\subsection{Citing and references (example)}

 Bibliography referencing requirements are described in the TUT
library reference list
\footnote{http://www.ttu.ee/institutes/library-3/for-students/reference-list/}.

\subsection{Colored temporary text}

\textcolor{red}{[Delete this note!]}

\subsection{Footnotes (example)}

Here is how you can use\footnote{In case you ever need to add a footnote.} footnotes.

\subsection{Colored box (example)}

\begin{tcolorbox}[arc=0pt, outer arc=0pt, boxrule=0pt, left=0mm]
\blindtext
\end{tcolorbox}
