\section{Thesis structure and requirements}

This section provides an overview of the requirements for thesis structure. The thesis consists of various required and optional parts always presented in the order specified herein. Table 2 lists the parts of the thesis document in the required order.


\begin{table}[!h]
\caption{Mandatory and optional parts of thesis in the required order of their presentation}
\centering
\begin{tabular}{|L{5.1cm}|L{9.1cm}|}
\hline
\textbf{Thesis part} &  \textbf{Conditions}\\  \hline

Title page & Required 

Thesis written in foreign language (e.g., English) have
2 title pages, the first in the language of thesis and the
second in Estonian\\ \hline

Author’s declaration of originality & Required (see 2.3 Author’s declaration of originality) \\ \hline
Thesis task page & Optional (see conditions in Section 2.3) \\ \hline
Abstract in thesis language & Required (see conditions in Section 2.5) \\\hline
Abstract in other languages & Required (see conditions in Section 2.5) \\\hline
List of abbreviations and terms & Required (see conditions in Section 2.6) \\\hline
Table of contents  & Required \\\hline
List of figures &  Optional; presented only if thesis contains figures \\\hline
List of tables  & Optional; presented only if thesis contains tables \\\hline
Introduction  &Required \\\hline
Chapters of thesis subject  development  &Required \\\hline
Summary  &Required \\\hline
References  &Required \\\hline
Appendices  &Optional; presented only if thesis contains any\\\hline
\end{tabular}
\end{table}

\newpage
\subsection{Title page}
The title page is formatted using the same font as the rest of the thesis, i.e., \textit{Times New
Roman} \footnote{It is allowed to use other serif fonts alike to \textit{Times New Roman} such as \textit{Times} for example}. The top of the page must include the name of the university and school. The location and the year of the thesis defence are presented at the bottom of the page. Title page is formatted using a table, as shown in Appendix 1. It has separate header and footer from the rest of the thesis. Author’s first, middle and last name, and student code are presented on separate lines. These are followed by thesis title and type (bachelor’s/master’s thesis). Supervisor(s) name(s) and detail(s) are presented after thesis
type. If necessary, consultants are listed as last.

Thesis written in English (or any other foreign language) has two (2) title pages: the first
in the language of thesis and the second in Estonian. Thesis written in Estonian has only
one title page.

\subsection{Page numbering}
All pages must be numbered. Numbering must be pervasive and include every page from the title page until the last page of appendices. Page numbers are presented at the bottom
of the page (in footer) and aligned centred using the style Footer or font 12pt. Page numbers are placed 1.5 cm from the bottom edge. There is no page number on title pages;
still they are counted in the numbering.

\subsection{Author’s declaration of originality}
Author’s declaration of originality is a required part of every thesis that follows the title page (Appendix 2). The statement of author’s declaration of originality is as follows:

\begin{tcolorbox}[arc=0pt, outer arc=0pt, boxrule=0pt, left=0mm]
I hereby certify that I am the sole author of this thesis and this thesis has not been
presented for examination or submitted for defence anywhere else. All used materials,
references to the literature and work of others have been cited.\\

Author: [First name Middle name(s) Last name] \\ \relax
[dd.mm.yyyy]
\end{tcolorbox}

\subsection{Thesis task specification}

The thesis must contain an explicit description of the problems the author is solving in
the thesis work.

For \textbf{Bachelor’s thesis} the task is formalized as a separate task specification document
(established by the department where the thesis will be defended; ask for the task sheet
form from your supervisor), or is specified in the Introduction section of the thesis
covering the following:
\begin{itemize}
\item Problems to be solved and initial conditions of the task,
\item Special conditions applicable for solving the specified problem (if any).
\end{itemize}
Filled task sheet form is added to the thesis as a separate page after Author’s declaration
of originality.
\textbf{For Master’s thesis} the task specification must be covered in thesis Introduction section
through the following points:
\begin{itemize}
\item  Problems to be solved and initial conditions of the task,
\item Special conditions applicable for solving the specified problem (if any).
\end{itemize}

\subsection{Thesis abstract}

Abstract is a mandatory part. It provides an overview about the aims, the most important
issues, results and conclusions of the thesis. Abstract is a short overview of thesis which
does not explain or justify the content but provides an overview of the most important
aspects. An abstract in Estonian is called Annotatsioon and in Russian \textcolor{red}{[Russian is not supported in this document yet!]}.
Depending on the thesis language the following abstracts must be included:
\begin{itemize}
\item If the thesis is written in English, the abstract is \sfrac{1}{2} A4 pages long and the abstract
in Estonian (Annotatsioon) is of length of 1 A4 page (except in case of the
graduation theses of degree curricula taught in English).

\item If the thesis is written in any other foreign language not English nor Estonian (e.g., Russian), the abstract in thesis language is \sfrac{1}{2} A4 pages long, which is followed by an abstract in Estonian (Annotatsioon) of length of 1 A4 page, and an abstract in English of length of 1 A4 page.
\end{itemize}

For abstracts not in the main thesis language, \textbf{thesis title in foreign language} is added in
between the heading Abstract and the abstract content. Abstracts in different languages
are presented on separate pages -- abstract heading is a level 1 heading which starts with
a new page.

The last paragraph of abstract is \textbf{obligatory} and must be written accordingly:

\begin{tcolorbox}[arc=0pt, outer arc=0pt, boxrule=0pt, left=0mm]
The thesis is in [language] and contains [pages] pages of text, [chapters] chapters,
[figures] figures, [tables] tables.
\end{tcolorbox}\vspace{-10pt}

The abstract in Estonian (Annotatsioon) has the following obligatory ending:

\begin{tcolorbox}[arc=0pt, outer arc=0pt, boxrule=0pt, left=0mm]
Lõputöö on kirjutatud [keel] keeles ning sisaldab teksti [lehekülgede arv töö põhiosas]
leheküljel, [peatükkide arv] peatükki, [Figurete arv] Figuret, [tabelite arv] tabelit.
\end{tcolorbox}\vspace{-10pt}

\textbf{NB!} Headings "Author’s declaration of originality", "Abstract", etc. are formatted as
unnumbered and centered level 1 headings (style \textit{Heading\_center}) and are not presented
in the table of contents. An example of Abstract is presented in Appendix 3.

\subsection{List of abbreviations and terms}


he list of abbreviations and terms must contain all new and ambiguous terms. For example the abbreviation PC can be used as Personal computer or Program counter. Abbreviations and terms are presented in a table of two columns where the left column contains terms or abbreviations and the right column provides their explanations. Foreign words are presented in italic. This table does not have any border lines. %(for table formatting in MS Word please refer to Section 3.5 and in LibreOffice Section 4.7).

The text in the list of abbreviations and terms table is formatted using the style \textit{Table\_text}.
Appendix 6 presents an example of this table layout \footnote{To add a new row to the table with the same formatting press TAB in the last cell of the table.} . The width of this table is the
maximum size of the content area; the distribution of the left and right column widths is
up to the author.

The heading of this section is level 1 heading starting with a new page. Use the style \textit{Heading\_center} for correct formatting.

Each abbreviation presented in the main text must be explained on the first occurrence.
For example, if an abbreviation PC is used, then it should be accompanied with explanation in parentheses, e.g. PC (\textit{Personal Computer}).

\subsection{Table of contents, list of figures and list of tables}

The table of contents lists the headings and corresponding start page numbers for all parts
of the thesis – sections (chapters) and subsections – starting from Introduction and ending
with Appendices. The table of contents is composed and structured according to headings.

If thesis is composed using the provided template and the author has used the heading styles as specified, the table of contents is generated automatically after an update is
performed on the “Table of Contents” page.

The list of figures presents only the figures available in the main part of the thesis -- figures
presented in appendices are not listed here. The heading "List of figures" starts with a new page. An example is given in Appendix 4.
 
The list of tables presents only the tables available in the main part of the thesis -- tables presented in appendices are not listed here. The heading "List of tables" starts with a new  page. An example is given in Appendix 5.
 
The lists of figures and tables are generated automatically after an update if the author has applied correct captions as specified in the template.

\subsection{Thesis content formatting}

This section addresses formatting requirement for the main parts of thesis starting with Introduction.

It is advised to start thesis composition by downloading the author’s toolkit and using the provided template. The templates contain ready-made styles that follow the requirements established in this document. Table 1 (page 6) provides an overview of available styles and their settings.


\subsubsection{Formatting the main parts of thesis}

In thesis Introduction the author presents the topic of the thesis, its goals, and highlights which problems will be solved. Also the division of the thesis subject development into chapters is presented. Introduction must contain starting conditions, subtasks and if needed -- additional requirements (see Section 2.4).

In thesis Summary the author presents the main goal of the thesis and provides answers to problems stated in the Introduction. Summary must also present the main results and important findings of the thesis.

The division of thesis content into sections and subsections has to be rational – sections
and subsections with only one paragraph of text must be avoided; they should be included
into some other section. The maximum depth of headings is three (3) levels, the use of
level 4 headings is not allowed.

In all sections and subsections the heading must be followed with text content. Authors are discouraged to add new heading, figure or table immediately after heading. Level 1
headings always start a new page, the space before the heading is 60 pt and after 18pt. Correct formatting of first level headings is ensured by exploitation of style Heading 1;
do note that different parts of thesis may have different requirements for headings (see Table \ref{Tab:formatting1}). The peculiarities of adding level 1 headings in MS Word version 2013 and newer are presented in Section 3, page \textcolor{red}{??}.


\subsubsection{Figures and tables}
All figures and tables in the thesis main part must be numbered and captioned with a title.
Numbering of figures and tables must be separately pervasive. Figure number and caption are added under the figure, table number and caption are added above the table. Figure
caption starts with a word "Figure" followed by a space, number, a dot, and figure title.
Table captions start with a word “Table” and follow the same formatting pattern. Figure/Table caption always ends with a dot.



Single line captions are aligned to center (use style Caption), whereas multiline captions are aligned as justify (use style Caption\_multiline). Figure/Table caption must be on the same page with the figure/table. Adding captions to figures and tables in word processing is described in more detail for MS Word in Section 3.6 and for LibreOffice Writer (LO)
in Section 4.6.

Figures and tables are placed into document separate from content text -- the content text is above and below figure/table and never on the left or right -- and centred. All figures and tables must be referred to in text and appropriate explanations provided. A reference to a figure or table within text must occur before the figure or table itself.

For figures and tables in language other than the thesis main language the original text may be preserved, if necessary it is presented in Italic. Abbreviations used on figures or in tables must be explained in caption title; if the abbreviation is used also elsewhere in the thesis, it must be added to the list of abbreviations and terms. The use of marking and symbols different from the thesis main content is allowed for figures.


An extra empty line of cells without borders is added at the end of table, to ensure enough spacing between the table and the following text. If a table spans across multiple pages,
the header row is duplicated at the beginning of each page (for an example see Table 1). It is recommended to avoid such tables whenever possible.

\begin{figure}[!ht]
\centering
\includegraphics[width=0.8\textwidth]{figures/TTU_peamine_logo_EST_print}
\caption{ Example: TUT main logo in Estonian}
\label{fig:logo}
\end{figure}

Figures (vector graphics) and illustrations (raster graphics) must be of printable quality. The lettering in figures should not use font sizes smaller than 6pt ($\sim$2mm character
height). The correct process of adding figures into thesis document using the styles in the template is described in Section 3.3 for MS Word and in LibreOffice in Section 4.6.
Figure \ref{fig:logo} presents a correctly formatted example of figure with short single line caption.

\begin{figure}[!ht]
\centering
\includegraphics[width=0.8\textwidth]{figures/TTU_alternatiivne_logo_EST_ENG_print}
\caption{Example: The alternative dual-language logo for Tallinn University of Technology. The logo is surrounded by a protected area (white space around) which guarantees its distinctiveness. Placing text or other graphical elements in the protected area is prohibited. The minimum height for the logo is 12 mm}
\label{fig:logo2}
\end{figure}

Figure \ref{fig:logo2} presents correctly formatted example of a figure with long multi-line caption.


For multi-line captions style Caption\_multiline is applied to or align Justify.

\begin{figure}[!ht]
\centering
	 \begin{subfigure}[b]{0.45\textwidth}
    \includegraphics[width=\textwidth]{figures/TTU_peamine_logo_ENG_print}
        \caption{}
    \end{subfigure}
    \qquad
    \begin{subfigure}[b]{0.45\textwidth}
    \includegraphics[width=\textwidth]{figures/ttu_peamine_logo_eng_must-valge_negatiivis}
        \caption{}
    \end{subfigure}
    \caption{Example: TUT main logo in: (a) English, (b) negative black and white}
    \label{fig:logo3}
\end{figure}

Figure \ref{fig:logo3} presents the correct formatting and referencing of different parts of the figure in
caption title in case of a multi-part figure.


\subsubsection{Program code}


Program code (source code) is formatted on the same principles as figures using monospaced font (for example Courier, Courier New or Consolas) with single line spacing. The template contains a defined style Program\_code for this purpose. Figure
\ref{alg:example} presents an example of formatted program code.

Program code must always be correctly intended. The code is presented continuously on the same page, whenever possible. This is achieved by style Program\_code and its application during formatting.

The first line of the program code starts from the left border. To obtain visually horizontal centered alignment, select the whole code and use the Increase Indent option.

Program code is captioned as figure.

\begin{algorithm}[!ht]
 Public Function computeSomething() \\
 Dim i, j As Integer\\
\For {i = 1 To 10}{
    \For {j = 1 To 10}{
    	Do something in loop\\
    Next j\\
    }
	Next i\\
    }
Return i + j\\
End Function\\
%use \captionof since algorithm should be declared as figure!
\captionof{figure}{How to write algorithms}\label{alg:example}
\end{algorithm}


\subsubsection{Mathematical expressions and equations}

Numbering mathematical equations is mandatory in case there is a reference to them in
the text. In all other cases numbering is optional. The number is placed on the same line
with the equation on the right side of the page in parentheses. Equation (1) presents an
example of correct formatting and referencing.

\begin{equation}
(x+a)^n = \sum_{k=0}^{n} {n \choose k} x^ka^{n-k}
\end{equation}

The tab stop for the number on the right is set to 14.8 cm, and is included in the style
Equation.


\subsubsection{Citing and references}

Numeric style should be used for references. There are two options to list references:
ordering references by author name and title in alphabetical order, or in the order they are cited in text. The list is numbered with consecutive Arabic numbers in square brackets, year of appearance is added at the end of the reference record. All used sources must be cited, including printed and electronic materials. The list of references must only include the sources, which have been actually used in the thesis, and which have been referenced in the text. When the list of references and the references in the text are non-compliant, it will call into question the merits of the work, as well as formatting accuracy. Bibliography referencing requirements are described in the TUT library reference list \footnote{http://www.ttu.ee/institutes/library-3/for-students/reference-list/} , which also
provides relevant examples.

For citations in the text, please use square brackets and Arabic numbers. The reference in
the text is always included into a sentence, either immediately after mentioning the source, at the end of the sentence (before the ending dot), or at the latest at the end of the
related topic paragraph’s last sentence. A reference must be placed to text at the first (suitable) chance, repeated citing is allowed. The reference number is never alone outside a sentence. Sentence never starts with a reference. Multiple references are presented in the ascending order each number in separate square brackets, e.g., [5], [7], [33], with a range of references a dash can be used e.g., [8]–[13]. When referring or citing a specific idea, sentence or data, a page number(s) must be added to the reference, e.g. [2, p 35]. Reference to a whole book, a book chapter, article, or other source can be used if the reference is based on the argument of the work as a whole.
Word processing MS Word includes a reference processing system to manage sources -- this is covered in Appendix 7.
If the number of referenced sources is not large, the numbered references list can be composed manually using the style List\_bibliogr included in the template.

\subsubsection{Appendices}
Appendices are used to present materials that add value to the main part of the thesis, help to understand a problem, provide additional details etc. The main text must contain a
reference for all added appendices. Appendices are not counted into the volume, and the use of them is optional.

Appendices are numbered in the ascending order (no automatic numbering of headings is used) with Arabic numbers or letters from the alphabet and formatted as first level
headings aligned to left (style Heading\_unnumber). Although the formatting of appendices is free, it is advisable to follow the guidelines and requirements applicable for
the main part of the thesis.

\subsubsection{Hints for text processing}

In order to take full advantage of the provided template, word processing program should be used in a correct manner. Word processing is not typing a text on a typewriter. The following list draws attention to some common errors made during word processing:

\begin{itemize}
\item Words are separated by only one space character between them. If a word is followed by a punctuation mark, punctuation follows the word immediately, and a space character is added after punctuation. Redundant spaces should be removed, for instance by search and replace.

\item Word processing automatically takes care of dividing text to lines. The division of text into paragraphs is done by author by pressing the Enter key on keyboard,
which adds a non-printable paragraph end mark ¶.

\item Sometimes it is necessary to keep words or symbols together so that they will not
be distributed on separate lines, for example initials of given name and surname. To add such a non-breaking space use keyboard combination Ctrl+Shift+Space.

\item To hold together words separated with hyphen, use nonbreaking hyphen, which can be added with keyboard combination Ctrl+Shift+hyphen.

\item \textbf{A dash} (--) is longer than hyphen and it is not the minus symbol. An easy way to
add dash into text is Ctrl+Numpad minus symbol, or Alt+0150.

\item Hyphenation should not be something to worry about. If you want to hyphenate a
word manually, just add Ctrl+Hyphen.

\item To break a long line (e.g., title), add a line break by pressing Shift+Enter.
\end{itemize}