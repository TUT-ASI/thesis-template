\documentclass{TalTechTemplate}
% --------------------------------------
% Some Guidelines for Using the template
% --------------------------------------
% 1. DO NOT MODIFY THE CLASS FILE (TalTechTemplate.cls)!
% 2. Try to keep the document modular (use `chapters` and other folders)
% --------------------------------------
%
% Recommendations:
% - subfigure package is deprecated! Please use subcaption package instead.
% - Use hhline instead of cline package


% Some useful packages, not necessary for the template:

\usepackage{blindtext}      % stubs
\usepackage{url}            % for using URLs
\usepackage{duckuments}     % image Stubs
\usepackage{placeins}       % setting FloatBarriers to force flushing of floating objects
\usepackage{float}          % sharp placement with H
\usepackage{color}          % used for \textcolor

\usepackage[table]{xcolor}  % colorful tables
\usepackage{multirow}       % for multi-row cells in table
\usepackage{hhline}         % for customizing the horizontal lines in tables
\usepackage{tcolorbox}      % for gray text boxes on pages 17 and 19

\usepackage{graphicx}       % figures inside text
\usepackage{subcaption}     % used for figures with sub figures

\usepackage{xfrac}          % for some fancy fraction numbers
\usepackage[inter-unit-product = \cdot]{siunitx} % correctly formatted SI units with defined sign between units

% footnotes in captions: https://tex.stackexchange.com/questions/154423/footnote-numbers-in-the-list-of-figures
\newcommand{\footcaption}[1]{\caption[#1]{#1\footnotemark}}


% use the following package if you need pseudo-code
% also, instead of using \caption please use:
% \captionof{figure}{[here is your caption!]}
% since algorithms should be declared as figure!
\usepackage[linesnumbered]{algorithm2e}
\SetAlFnt{\small\bfseries\ttfamily}


% Bibliography management through Biblatex. *.bib database created
% with tools such as Mendeley.
% biblatex author year style:
% https://tex.stackexchange.com/questions/261430/biblatex-author-year-citations
\usepackage{csquotes}
\usepackage[backend=biber,
            sorting=none,
            style=ieee,
            ]{biblatex} % authoryear,
% separate author year by comma for \parencite{}
\renewcommand*{\nameyeardelim}{\addcomma\addspace}
\addbibresource{literature.bib}
% \cite -> author year ; \textcite{key} -> author (year); \parencite{key} -> (author year)

% Glossaries Management: https://en.wikibooks.org/wiki/LaTeX/Glossary
% https://tex.stackexchange.com/questions/300049/remove-empty-pages-before-glossary/300058


\usepackage[nopostdot, toc, nonumberlist, acronym, section=section]{glossaries}
\renewcommand*{\glsclearpage}{} % to prevent empty page before glossary
\makenoidxglossaries
% Load Glossary file
\loadglsentries{preliminary/glossaries}


% Edit your information here!

% Choose the language here:
\mainlanguage{english}
%\mainlanguage{estonian}


% Name of the author
\authorname{[First name Middle name Last name]}


% Thesis title (English)
\worktitle{[Title of the Thesis]}
% Thesis title (Estonian)
\worktitleEst{[Lõputöö pealkiri]}


% Supervisor's name
\supervisorname{[Supervisor's Name]}
% Supervisor's title (English)
\supervisortitle{[Supervisor's Title]}
% Supervisor's title (Estonian)
\supervisortitleEst{[Teaduskraad]}
% Co-supervisor's name
\cosupervisorname{[Co-supervisor's Name]}
% Co-supervisor's title (English)
\cosupervisortitle{[Co-Supervisor's Title]}
% Co-supervisor's title (Estonian)
\cosupervisortitleEst{[Teaduskraad]}


% "Code of graduation thesis"
% You can see it when you submit an application
% in ÕIS (type of application: "Defence of graduation thesis")
\programcode{[IAY70LT]}
% Student code
\studentcode{[123456ABCD]}


% School (English)
\school{School of Information Technologies}
% School (Estonian)
\schoolEst{Infotehnoloogia teaduskond}
% Department (English)
\department{Department of Computer Systems}
% Department (Estonian)
\departmentEst{Arvutisüsteemide instituut}


\begin{document}
\sloppy % to make sure mbox does not go over the limit

\prologue % title pages, abstracts, etc.

% ----------------------------- INTRODUCTION ------------------------------
\section{%
  \IfLanguageName{english}{Introduction}{}%
  \IfLanguageName{estonian}{Sissejuhatus}{}%
}
\label{Introduction} % Allows referencing titles with \ref

[Text]


% ----------------------------- CHAPTERS ----------------------------------
\input{chExamples1/chapter1.tex}

\section{[Chapter Title]}

This chapter provides some essential \LaTeX\ examples (lists, tables,
etc.) and hints.

\subsection{Hints for text processing}

Here are some tips for using \LaTeX:

\begin{itemize}
\item Words are separated by one or       more space characters between
  them. If a word is followed by a punctuation mark, punctuation should
  follow the word immediately, and a space character should be added
  after punctuation. Redundant
  spaces      are
  automatically             ignored.

\item Text is automatically divided into lines. The division of text
  into paragraphs is done by using two newlines.

\item Sometimes it is necessary to keep words or symbols together so
  that they will not be distributed on separate lines, for example
  initials of given name and surname. To add such a non-breaking space
  use \textasciitilde\ character (example: Kersti~Kaljulaid).

\item To prevent other words from breaking you can use mbox. Note
  that using \mbox{non-breaking} hyphen from extdash package is also
  possible, but it will not prevent the same word to be broken
  elsewhere.

\item \textbf{A dash} (--) is longer than hyphen and it is not just
  the minus symbol. Use two minus characters to insert a dash
  (example: --).

\item Hyphenation should not be something to worry about. If you want to hyphenate a
  word manually, just use \verb!\hyphenation{every-where}! in the document preamble.

\end{itemize}

\subsection{Unordered list (example)}

Unordered list example:
\begin{itemize}
\item Item in unordered list
\item Deeper unordered list:
  \begin{itemize}
  \item \textbf{Foo}
  \item \textit{Bar}
  \end{itemize}
\end{itemize}


\subsection{Citation with Bib\LaTeX \& Biber}
To manage the references aka. bibliography the \LaTeX package \emph{biblatex} is used. Furthermore, the backend biber is used to have the full control of style and further parameters. This can be achieved by calling \textbackslash usepackage[backend=biber, sorting=none, style=ieee]\{biblatex\}. The style can be changed to any supported one by biblatex. The parameter sorting \emph{none} arranges the reference in the first appearance order. The actual bibliography is loaded by calling \emph{\textbackslash addbibresource\{literature.bib\}}.\par
For studies outside of the IEEE domain the Author/Year representation might be more common and can be activated by setting the style to \emph{authoryear}. Therefore, a delimiter has to be set for correct representation which can be achieved with the following command. \emph{\textbackslash renewcommand*\{\textbackslash nameyeardelim\}\{\textbackslash addcomma\textbackslash addspace\}}.\par
To successfully generate the bibliography \LaTeX has to be ran multiple times including one bibtex run between. The compile sequence is as follow: \LaTeX - Biblatex - \LaTeX - \LaTeX \par
For managing the references, tools such as Mendeley are strongly recommended. Reference management tools allowing one to manually add or download references from an online repository and to manage them. Furthermore, the references can be managed in groups and backed-up into the cloud. Once done, a bib file can be generated which then is loaded as described above.


\subsubsection{Citation (example)}

Example of citation \cite{limewiki}.\\
As shown by \textcite{Zavorotny2013} .... Later on, it was shown in a experiment \parencite{Hobiger2014} that ...


\subsection{Terms \& Abbreviations}
\gls{sdr}, \gls{sdr}, \glspl{sdr}, \glsreset{sdr}\glspl{sdr}\\
\gls{zenit}, \glspl{ddm}

\newpage
\subsection{Table (example)}

\begin{table}[!h]
  \caption{This is a table caption}
  \centering
  \begin{tabular}{|L{5.1cm}|L{9.1cm}|}
    \hline
    \textbf{Column 1} & \textbf{Column 2}\\ \hline
    One row     & Some text\\ \hline
    Another row & Single-line cell \\

                & More text here\\ \hline

    Foo & Text here\\ \hhline{-~}
    Bar & (more more)\\ \hhline{-~}
    Baz & \\ \hline
  \end{tabular}
\end{table}

\newpage
\subsection{Fancy Table (example)}

\begin{table}[!h]
  \caption{This is another table caption}
  \centering
  \begin{tabular}{|L{5.1cm}|L{9.1cm}|}
    \hline
    \multicolumn{2}{|c|}{\textbf{Multi-Column Example}}\\\hline
    \cellcolor{blue!25} \textbf{Cell color example}     & Some text\\ \hline
    \multirow{2}{*}{\textbf{Multi-row Example}} & Single-line cell \\
                                                & More text here   \\ \hline
    Foo & Text here\\ \hhline{-~}
    Bar & (more more)\\ \hhline{-~}
    Baz & \\ \hline
  \end{tabular}
\end{table}

\subsection{Figures (example)}

% TODO use public domain images

\begin{figure}[!ht]
  \centering
  \includegraphics[width=0.8\textwidth]{chExamples2/figures/TTU_peamine_logo_EST_print}
  \caption{ Example: TUT main logo in Estonian}
  \label{fig:logo}
\end{figure}

Figure \ref{fig:logo} presents an example of figure with short single
line caption.

\begin{figure}[!ht]
  \centering
  \includegraphics[width=0.8\textwidth]{chExamples2/figures/TTU_alternatiivne_logo_EST_ENG_print}
  \caption{Example: Lorem ipsum dolor sit amet, consectetuer
    adipiscing elit. Etiam lobortis facilisis sem. Nullam nec mi et
    neque pharetra sollicitudin. Praesent imperdiet mi nec ante.
  }
  \label{fig:logo2}
\end{figure}

Figure \ref{fig:logo2} presents an example of a figure with long multi-line caption.


\begin{figure}[!ht]
  \centering
  \begin{subfigure}[b]{0.45\textwidth}
    \includegraphics[width=\textwidth]{chExamples2/figures/TTU_peamine_logo_ENG_print}
    \caption{}
  \end{subfigure}
  \hfill
  \begin{subfigure}[b]{0.45\textwidth}
    \includegraphics[width=\textwidth]{chExamples2/figures/ttu_peamine_logo_eng_must-valge_negatiivis}
    \caption{}
  \end{subfigure}
  \caption{Example: TUT main logo in: (a) English, (b) negative black and white}
  \label{fig:logo3}
\end{figure}

Figure \ref{fig:logo3} presents an example of a multi-part figure.


\FloatBarrier
\subsection{Figures, Captions \& Footnotes}
\subsubsection{Figures with Minipages}
\begin{figure}[H]
	\centering
	\begin{minipage}[b]{0.4\textwidth}
		\includegraphics[width=\textwidth]{example-image-duck}
		\caption{Flower one.}
	\end{minipage}
	\hfill
	\begin{minipage}[b]{0.4\textwidth}
		\includegraphics[width=\textwidth]{example-image-duck}
		\caption[Flower two, caption in list of figures]{Flower two, caption under figure}
	\end{minipage}
\end{figure}

\subsubsection{Figures with SubFigures}
% Example source: https://tex.stackexchange.com/questions/196481/problem-on-subfigure-2x2
\begin{figure}[H]
	\centering
	\begin{subfigure}[b]{0.475\textwidth}
		\centering
		\includegraphics[width=\textwidth]{example-image-duck}
		\caption{{\small Duck 1}}
		\label{fig:duck1}
	\end{subfigure}
	\hfill
	\begin{subfigure}[b]{0.475\textwidth}  
		\centering 
		\includegraphics[width=\textwidth]{example-image-duck}
		\caption{{\small Duck 2}}
		\label{fig:duck2}
	\end{subfigure}
	\vskip\baselineskip
	\begin{subfigure}[b]{0.475\textwidth}
		\centering
		\includegraphics[width=\textwidth]{example-image-duck}
		\caption{{\small Duck 3}}
		\label{fig:duck3}
	\end{subfigure}
	\quad
	\begin{subfigure}[b]{0.475\textwidth}
		\centering
		\includegraphics[width=\textwidth]{example-image-duck}
		\caption{{\small Duck 4}}
		\label{fig:duck4}
	\end{subfigure}
	\footcaption{Duck Test, figure caption text}
	\label{fig:ducks}
\end{figure}
\footnotetext{\href{https://www.google.com}{www.google.com is great}, last visited \today}


\subsection{SI Units}
\SI{2.012}{\newton\metre}, \SI{3.412}{\newton\square\meter\per\cubic\kilogram}


\subsection{Program code}

Figure \ref{alg:example} presents an example of formatted program
code.

\begin{algorithm}[!ht]
  Public Function computeSomething() \\
  Dim i, j As Integer\\
  \For {i = 1 To 10}{
    \For {j = 1 To 10}{
      Do something in loop\\
      Next j\\
    }
    Next i\\
  }
  Return i + j\\
  End Function\\
  % use \captionof since algorithm should be declared as figure!
  \captionof{figure}{How to write algorithms}\label{alg:example}
\end{algorithm}


\subsection{Mathematical expressions and equations (example)}

Equation \ref{eq:equation-example} presents an example of correct
formatting and referencing.

\begin{equation} \label{eq:equation-example}
  (x+a)^n = \sum_{k=0}^{n} {n \choose k} x^ka^{n-k}
\end{equation}


\subsection{Colored temporary text}

\textcolor{red}{[Delete this note!]}

\subsection{Footnotes (example)}

Here is how you can use\footnote{In case you ever need to add a footnote.} footnotes.

\subsection{Colored box (example)}

\begin{tcolorbox}[arc=0pt, outer arc=0pt, boxrule=0pt, left=0mm]
\blindtext
\end{tcolorbox}


% ----------------------------- SUMMARY -----------------------------------
\section{%
  \IfLanguageName{english}{Summary}{}%
  \IfLanguageName{estonian}{Kokkuvõte}{}%
}
\label{Summary}

\blindtext


% ----------------------------- BIBLIOGRAPHY ------------------------------
% *** Set References
% https://www.sharelatex.com/learn/Bibliography_management_with_bibtex
\footnotesize
\setstretch{0}
%\normalem	% surpress underlines in bibliography
\printbibliography[heading=bibintoc, title={References}]

\epilogue{
  \input{appendix/appendix1.tex}
  \input{appendix/appendix2.tex}
}
\end{document}


%%% Local Variables:
%%% mode: latex
%%% TeX-master: t
%%% End:
