\documentclass{MasterThesisTUT}

\newcommand{\ThesisInEnglish}{true}	% set to true if in English
\usepackage{ifthen}
% --------------------------------------
% Some Guidelines for Using the template
% --------------------------------------
% 1. DO NOT MODIFY THE CLASS FILE!
% 2. DO NOT MODIFY THE CLASS FILE!
% 3. Try to keep the document modular (use `chapters` and `figures` folders)
% ----------------------------------
% Recommendations:
% - subfigure package is deprecated! Please use subcaption package instead.
% - Use hhline instead of cline package


% some useful packages, not necessary for the template:
\usepackage[table]{xcolor}
\usepackage{graphicx}       % Figures inside text
\usepackage{url}            % For using URLs
\usepackage{blindtext}      % Stubs
\usepackage{color}          % used for \textcolor
\usepackage{multirow}       % for multi-row cells in table
\usepackage{hhline}         % for customizing the horizontal lines in tables
\usepackage{tcolorbox}      % for gray text boxes on pages 17 and 19
\usepackage{xfrac}          % for some fancy fraction numbers used in the document
\usepackage{subcaption}     % used for figures with sub figures a,b, etc
\usepackage[iso]{datetime}  % current date (note: uses ISO 8601 format)


% use the following package if you need pseudo-code
% also, instead of using \caption please use:
% \captionof{figure}{[here is your caption!]}
% since algorithms should be declared as figure!
\usepackage[linesnumbered]{algorithm2e}
\SetAlFnt{\small\bfseries\ttfamily}

% Bibliography management through Biblatex. *.bib database created with tools such as Mendeley.
% biblatex author year style: https://tex.stackexchange.com/questions/261430/biblatex-author-year-citations
\usepackage{csquotes}
\usepackage[backend=biber, sorting=none, style=ieee]{biblatex}	% authoryear, 
\renewcommand*{\nameyeardelim}{\addcomma\addspace}	% separate author year by comma for \parencite{}
\addbibresource{literature.bib}
% \cite -> author year ; \textcite{key} -> author (year); \parencite{key} -> (author year)

% Glossaries Management: https://en.wikibooks.org/wiki/LaTeX/Glossary
% https://tex.stackexchange.com/questions/300049/remove-empty-pages-before-glossary/300058
%\usepackage[acronym, nonumberlist, toc]{glossaries}
\usepackage[nopostdot, toc, nonumberlist, acronym, section=section]{glossaries}
\renewcommand*{\glsclearpage}{}	% to prevent empty page before glossary
\makenoidxglossaries
% Load Glossary file
\loadglsentries{preliminary/glossaries}

% -------------------------------
% ⚠ Edit your information here! ⚠
% -------------------------------
\authorname{[First name Middle name Last name]} % Name of the author
\thesistitle{[Title of the Thesis]}         % Title of Thesis (English)
\estonianthesistitle{[Lõputöö pealkiri]}    % Title of Thesis (Estonian)
\supervisorname{[Supervisor's Name]}        % Name of the supervisor
\supervisortitle{[Supervisor's Title]}      % Supervisor's title
\cosupervisorname{[Co-supervisor's Name]}   % Name of the co-supervisor
\cosupervisortitle{[Co-Supervisor's Title]} % Co-supervisor's title
\programcode{[IAY70LT]}
\studentcode{[1234ABCD]}
\school{School of Information Technologies}
\department{Department of Computer Systems}
\estonianschool{Infotehnoloogia teaduskond}
\estoniandepartment{Arvutisüsteemide instituut}

\begin{document}
\sloppy % to make sure mbox does not go over the limit

% Document structure:

% ----------------------------- TITLE PAGE --------------------------------
\maketitlepage

% ----------------------------- TITLE PAGE (EST) --------------------------
\maketitlepageeesti

% ----------------------------- DECLARATION -------------------------------
\section*{\centering Author's declaration of originality}
I hereby certify that I am the sole author of this thesis. All the
used materials, references to the literature and the work of others
have been referred to. This thesis has not been presented for
examination anywhere else.

Author: \nameofauthor

% TODO reference doc recommends [dd.mm.yyyy], but maybe use yyyy-mm-dd
\today
\pagebreak


\ifthenelse{\equal{\ThesisInEnglish}{true}}{
% ----------------------------- ABSTRACT ----------------------------------
\section*{\centering Abstract}

[Text]

This thesis is written in [language] and is \pageref{EndOfMainPart}
pages long, including [number] chapters, [number] figures and [number]
tables.

\pagebreak

}{
% ----------------------------- ABSTRACT (In Estonian) --------------------
\section*{\centering Annotatsioon \\ \estoniantitleofthesis}

[Tekst]

Lõputöö on kirjutatud [mis keeles] keeles ning sisaldab teksti
\pageref*{EndOfMainPart} leheküljel, \total{section} peatükki,\linebreak
\total{figure} joonist, \total{table} tabelit.

\pagebreak

}
% ----------------------------- ABBREVIATIONS -----------------------------
{\titleformat{\section}[display]{\Large\bfseries\centering}{}{}{}
\printnoidxglossary[title={List of Terms}]
\printnoidxglossary[type=acronym, title={List of Abbreviations}]
}

% ----------------------------- TABLE OF CONTENTS -------------------------
\singlespacing
\tableofcontents
\newpage
% ----------------------------- LIST OF FIGURES ---------------------------
\listoffigures
\pagebreak
% ----------------------------- LIST OF TABLES ----------------------------
\listoftables
\pagebreak
\onehalfspacing
% ----------------------------- INTRODUCTION ------------------------------
\section{Introduction}
\label{Introduction} % Allows referencing titles with \ref

[Text]

% ----------------------------- CHAPTERS ----------------------------------
\section{[Heading of Chapter]}

[Text]

% You can write comments by using % character

% Feel free to insert page breaks by using \newpage

\subsection{[Subheading]}

[Text]

\subsubsection{[Subsubheading]}

[Text]


\section{[Chapter Title]}

This chapter provides some essential \LaTeX examples (lists, tables,
etc.) and hints.

\subsection{Hints for text processing}

Here are some tips for using \LaTeX:

\begin{itemize}
\item Words are separated by one or       more space characters between
  them. If a word is followed by a punctuation mark, punctuation should
  follow the word immediately, and a space character should be added
  after punctuation. Redundant
  spaces      are
  automatically             ignored.

\item Text is automatically divided into lines. The division of text
  into paragraphs is done by using two newlines.

\item Sometimes it is necessary to keep words or symbols together so
  that they will not be distributed on separate lines, for example
  initials of given name and surname. To add such a non-breaking space
  use \textasciitilde\ character (example: Kersti~Kaljulaid).

\item To prevent other words from breaking you can use mbox. Note
  that using \mbox{non-breaking} hyphen from extdash package is also
  possible, but it will not prevent the same word to be broken
  elsewhere.

\item \textbf{A dash} (--) is longer than hyphen and it is not just
  the minus symbol. Use two minus characters to insert a dash
  (example: --).

% TODO ↓ what's the LaTeX equivalent for this?
\item Hyphenation should not be something to worry about. If you want to hyphenate a
  word manually, just add Ctrl+Hyphen.

% TODO ↓ is this correct?
\item To break a long line (e.g., title), use double backslash (example: hello\\world).
\end{itemize}

\subsection{Unordered list (example)}

Unordered list example:
\begin{itemize}
\item Item in unordered list
\item Deeper unordered list:
  \begin{itemize}
  \item \textbf{Foo}
  \item \textit{Bar}
  \end{itemize}
\end{itemize}

\subsection{Citation (example)}

Example of citation \cite{urlSource}.

\newpage
\subsection{Table (example)}

\begin{table}[!h]
  \caption{This is a table caption}
  \centering
  \begin{tabular}{|L{5.1cm}|L{9.1cm}|}
    \hline
    \textbf{Column 1} & \textbf{Column 2}\\ \hline
    One row     & Some text\\ \hline
    Another row & Multi-line cell

                  More text here\\ \hline

    Foo & Text here\\ \hhline{-~}
    Bar & (more more)\\ \hhline{-~}
    Baz & \\ \hline
  \end{tabular}
\end{table}

\newpage
\subsection{Fancy Table (example)}

% TODO take some example from Siavoosh

\subsection{Figures (example)}

% TODO use public domain images

\begin{figure}[!ht]
  \centering
  \includegraphics[width=0.8\textwidth]{figures/TTU_peamine_logo_EST_print}
  \caption{ Example: TUT main logo in Estonian}
  \label{fig:logo}
\end{figure}

Figure \ref{fig:logo} presents an example of figure with short single
line caption.

\begin{figure}[!ht]
  \centering
  \includegraphics[width=0.8\textwidth]{figures/TTU_alternatiivne_logo_EST_ENG_print}
  \caption{Example: Lorem ipsum dolor sit amet, consectetuer
    adipiscing elit. Etiam lobortis facilisis sem. Nullam nec mi et
    neque pharetra sollicitudin. Praesent imperdiet mi nec ante.
  }
  \label{fig:logo2}
\end{figure}

Figure \ref{fig:logo2} presents an example of a figure with long multi-line caption.


\begin{figure}[!ht]
  \centering
  \begin{subfigure}[b]{0.45\textwidth}
    \includegraphics[width=\textwidth]{figures/TTU_peamine_logo_ENG_print}
    \caption{}
  \end{subfigure}
  \qquad
  \begin{subfigure}[b]{0.45\textwidth}
    \includegraphics[width=\textwidth]{figures/ttu_peamine_logo_eng_must-valge_negatiivis}
    \caption{}
  \end{subfigure}
  \caption{Example: TUT main logo in: (a) English, (b) negative black and white}
  \label{fig:logo3}
\end{figure}

Figure \ref{fig:logo3} presents an example of a multi-part figure.

\subsection{Program code}

Figure \ref{alg:example} presents an example of formatted program
code.

\begin{algorithm}[!ht]
  Public Function computeSomething() \\
  Dim i, j As Integer\\
  \For {i = 1 To 10}{
    \For {j = 1 To 10}{
      Do something in loop\\
      Next j\\
    }
    Next i\\
  }
  Return i + j\\
  End Function\\
  % use \captionof since algorithm should be declared as figure!
  \captionof{figure}{How to write algorithms}\label{alg:example}
\end{algorithm}


\subsection{Mathematical expressions and equations (example)}

Equation \ref{eq:equation-example} presents an example of correct
formatting and referencing.

\begin{equation} \label{eq:equation-example}
  (x+a)^n = \sum_{k=0}^{n} {n \choose k} x^ka^{n-k}
\end{equation}


\subsection{Citing and references (example)}

 Bibliography referencing requirements are described in the TUT
library reference list
\footnote{http://www.ttu.ee/institutes/library-3/for-students/reference-list/}.

\subsection{Colored temporary text}

\textcolor{red}{[Delete this note!]}

\subsection{Footnotes (example)}

Here is how you can use\footnote{In case you ever need to add a footnote.} footnotes.

\subsection{Colored box (example)}

\begin{tcolorbox}[arc=0pt, outer arc=0pt, boxrule=0pt, left=0mm]
\blindtext
\end{tcolorbox}


% ----------------------------- CONCLUSION --------------------------------
\section{Summary}
\label{Summary} 

\blindtext

\pagebreak

% ----------------------------- BIBLIOGRAPHY ------------------------------
% *** Set References
% https://www.sharelatex.com/learn/Bibliography_management_with_bibtex
\footnotesize
\setstretch{0}
%\normalem	% surpress underlines in bibliography
\printbibliography[heading=bibintoc, title={References}]

\label{EndOfMainPart}
\newpage

%todo: find if we can somehow move this to the preamble!
\let\svaddcontentsline\addcontentsline
\renewcommand\addcontentsline[3]{%
  \ifthenelse{\equal{#1}{lof}}{}%
  {\ifthenelse{\equal{#1}{lot}}{}{\svaddcontentsline{#1}{#2}{#3}}}}


\begin{appendices}
 \setcounter{table}{0}
 \setcounter{figure}{0}
  \titleformat{\section}{\bfseries\Large}{Appendix \thesection}{0.2em}{}
  \renewcommand{\thesection}{\arabic{section} --}

  \section{[Heading of Appendix]}
Program code example:

\begin{algorithm}[!ht]
  Public Function computeSomething() \\
  Dim i, j As Integer\\
  \For {i = 1 To 10}{
    \For {j = 1 To 10}{
      Do something in loop\\
      Next j\\
    }
    Next i\\
  }
  Return i + j\\
  End Function\\
  % use \captionof since algorithm should be declared as figure!
  \captionof{figure}{How to write algorithms}\label{alg:example2}
\end{algorithm}

  \section{[Heading of Appendix]}
This template can be found on:
\begin{itemize}
\item \url{https://github.com/TUT-ASI/thesis-template}
\item \url{https://www.overleaf.com/read/wjghvfmyhwvc}
\end{itemize}


If you want to contribute or want to report bugs, please
\href{https://github.com/TUT-ASI/thesis-template/issues}{file an issue}
on GitHub or simply send an email to $siavoosh.azad@ttu.ee$.

\end{appendices}

\addtocounter{section}{-1} % Introduction
\addtocounter{section}{-1} % Summary

\end{document}
